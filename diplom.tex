%комментарии все для меня, чтобы я помнила, что к чему :)
\documentclass[a4paper,12pt]{report}[2015\03\20]
\usepackage[T2A]{fontenc}
\usepackage[utf8]{inputenc} %кодировка
\usepackage[english,russian]{babel} %расознование языка
\usepackage[dvips]{graphicx} %работа с картинками
\usepackage{geometry} %способ ручной установки полей  

\geometry{top=2cm} %поле сверху  
\geometry{bottom=2.5cm} %поле снизу  
\geometry{left=2.5cm} %поле справа  
\geometry{right=2cm} %поле слева
%\righthyphenmin=2 перенос содно строки на другую
\setcounter{tocdepth}{3}
\renewcommand
\contentsname{Содержание}
\begin{document}
\begin{titlepage}
\newpage

\begin{center}
Министерство образования и науки РФ \\
Федеральное государственное автономное образовательное учреждение высшего профессионального образования\\*
\vspace{1cm}
КАЗАНСКИЙ ФЕДЕРАЛЬНЫЙ (ПРИВОЛЖСКИЙ) УНИВЕРСИТЕТ \\
ИНСТИТУТ ВЫЧИСЛИТЕЛЬНОЙ МАТЕМАТИКИ И ИНФОРМАЦИОННЫХ ТЕХНОЛОГИЙ \\*
Кафедра технологии программирования
\end{center}

\vspace{8em}

\begin{center}
\Large Дипломная работа \\
на тему:
\end{center}

\vspace{1.5em}
 
\begin{center}
\textsc{\Large{\textbf{Использование методик и подходов коллективной разработки в нетехнических специальностях и отраслях}}}
\end{center}

\vspace{10em}
 
\begin{flushleft}
\textbf{Работу выполнила:} \\*
Студентка гр. 09-108 \hrulefill Кадирова А.М. \\
\vspace{1.5em}
Научный руководитель \hrulefill Туйкин А.М.\\
\vspace{1.5em}

\end{flushleft}
 
\vspace{\fill}

\begin{center}
Казань \\ 2015
\end{center}

\end{titlepage} 

\tableofcontents %оглавление. Должно создаваться автомотически.


\chapter*{Введение}
\begin 
\hspace{\parindent}
\indent На сегодняшний день информационные технологии очень тесно переплелись и продолжают переплетаться с нашей повседневной жизнью, а инструменты, подходы и практики, зародившиеся в мире программистов, находят применение в иных сферах. Однако, есть некоторые инструменты, которые созданы и используются сугубо программистами. Хотя методики и подходы могут быть эффективно использованы и людьми, не имеющими прямого отношения к процессу разработки. 

\indent Одной из такой сфер является командная работа.
Актуальность темы обусловлена тем, что современные реалии демонстрируют, что вопрос организации процесса в командах остро стоит не только в командах, занимающихся программной разработкой. Один из вопросов - как организовать процесс распределенных команд. Исходя из того, что в командах разработки есть положительный опыт в организации такого процесса, возникает идея переноса наработанных подходов в другие сферы. 

\indent Объектом исследования данной работы являются системы версионного контроля.

\indent Предметом исследования являются методологии и подходы в процессе командной разработки.

\indent Цель исследования состоит в попытке провести аналогии между процессами, положительно зарекомендовавшими себя в командной разработки ПО и переноса их в контекст другой предметной области.
Для достижения указанной цели в курсовой работе решаются следующие исследовательские задачи:

\begin{itemize}
\item	Исследовать процессы взаимодействия и инструменты на примере такой задачи, как создание совместной издательской работы (книги, результаты исследований, методологические пособия).
\item	Исследовать процесс и инструменты работы в команде на примере системы контроля версий git.
\item	Провести аналогии и попробовать перенести подходы, используемые в git на другую предметную область.
\item	Выявить плюсы и минусы. Предложить решения.

\end{itemize}
\end

\begin{thebibliography}{00}
\bibitem{chacon:progit} Scott Chacon.
\emph{<<Pro Git>>}. 2009;
\bibitem{herrero:markdown} Arturo Herrero.
\emph{<<Instant Markdown>>}.
Published by Packt Publishing Ltd.,
august 2013.
\bibitem{Lynn:gitMagic} Ben Lynn.
\emph{<<Git Magic>>}. 2010
\bibitem{Baldin:Latex} Балдин Е.М.
\emph{<<Компьютерная типография LaTEX>>},
/издательство «БХВ-Петербург», 2008
\bibitem{Leonov:clouds} Леонов В.М. 
\emph{<<Google Docs, Windows Live и другие облачные технологии>>},
/ООО «Издательство «Эксмо», 2012;
\bibitem{l'vovskii:verstka} Львовский С.М.  
\emph{<<Набор и вёрстка в системе LaTEX>>},
/3-е издание, исправленное и дополненное, 2003;
\end{thebibliography}

\end{document}
